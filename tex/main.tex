\documentclass{article}
\usepackage{fullpage}
\usepackage{parskip}
\usepackage{standalone}
\usepackage{graphicx}
\usepackage{booktabs}
\usepackage{subcaption}
\usepackage{hyperref}
\usepackage{amsmath}
\usepackage{amsfonts}
\usepackage{amsthm}
\usepackage[table]{xcolor}
\usepackage{tikz}
\usetikzlibrary{arrows}
\usetikzlibrary{decorations.pathreplacing}

\newtheorem{theorem}{Theorem}


\title{Modelling Queues Where Customers Randomly  Change Priority Classes While Waiting}
\author{Geraint I. Palmer, Michalis Panayidis \& Vincent Knight}
\date{}

\begin{document}
\maketitle

\section{Introduction}
There are a number of situations in which a customer's priority in a queue might
change during their time queueing, or equivalently where their priority depends
on the amount of time they have already spent in the queue.
Classic examples arise in healthcare systems, for example when a patient's
medical urgency might increase the longer they spend waiting due to health
degeneration. Another example would be a prioritisation scheme that attempts a
trade-off between medical need and waiting times.
These are both examples where a patient's priority has the chance to upgrade
over time while in the queue.
However there also might be situations in which a patient's priority can
downgrade over time: consider a medical intervention that can improve a
patient's outcome if caught early, if a patient has been waiting a long time
already then they might be passed over for a newly referred patient who will
gain more benefit from the intervention. In this case a patient's priority is
downgraded the longer they wait.

In this paper a single $M/M/c$ queue is modelled, with multiple classes of
customer of different priorities. While waiting in the queue customers change
their class to any other class at specific rates. Thus upgrades, downgrades, and
`skip-grades' (moving to a priority class not immediately above or below the
current class) are modelled.

This is first modelled using simulation, where we describe generalisable logic.
This is implemented in version v2.3.0 of the Ciw library in Python
\cite{palmer19}.
Then two Markov chain models are defined for the system, which are used to find
steady state distributions and expected sojourn times for each customer class.
These Markov chains give some insights into the behaviour of the systems under
different combinations of parameters; and numerical experiments give further
behaviours.

Systems of this kind have been investigated previously:

\begin{itemize}
  \item \cite{jackson60} (1960): Non preemptive M/M/1 where customers
      are served in order of the difference between their waiting time
        and urgency number (that is priorities increasing linearly over
        time). Solved by considering event probabilities at clock ticks.
  \item \cite{holtzman71} (1971): Similar to the above, but treat each
      urgency number as a separate customer class, and not considering
        clock ticks. Upper and lower bounds on the waiting times, based
        on FIFO and static priorities.
  \item \cite{netterman79} (1979): Now considers the case where
      priorities increase non-linearly but concavely over time.
  \item \cite{fratini90} (1990): Non preemptive M/G/1 queue with two
      classes of customers, where priorities switch if the number from
        one class exceeds a given threshold. Lower priority customers
        have a finite waiting capacity, higher have infinite capacity.
  \item \cite{knessl03} (2003): Similar to the above but with Markovian
      services and infinite waiting capacities for both customers.
  \item \cite{xie08} (2008): Preemptive n-priority-classes M/M/c with
      exponential upgrades. Customers only upgrade to the priority
        immediately higher than themselves. Stability considered.
  \item \cite{down10} (2010): Preemptive two-priority-classes M/M/c with
      exponential upgrades. Customers cannot upgrade if the number of
        lower priority customers is below a given threshold. Holding
        costs considered.
  \item \cite{he12} (2012): Extension of the above, allows batch
      arrivals, multiple classes, phase-type upgrades and services.
        Customers only upgrade to the priority immediately higher than
        themselves.
  \item \cite{bilodeau22} (2022): Analytical (truncated) expressions for
      a two class delayed accumulating priority M/G/1 queue. Customer
        priorities increase linearly over time, at different rates
        according to class, after an initial fixed delay.
\end{itemize}



\section{The Models}\label{sec:model}
Consider an $M/M/c$ queue with $K$ classes of customer labelled
$0, 1, 2, \dots, K-1$.
Let:

\begin{itemize}
  \item $\lambda_k$ be the arrival rate of customers of class $k$,
  \item $\mu_k$ be the service rate of customers of class $k$,
  \item $\theta_{i,j}$ be the rate at which customers of class $i$ change
  to customers of class $j$ while they are waiting in line.
\end{itemize}

Customers of class $i$ have priority over customers of class $j$ if $i < j$.
Customers of the same class are served in the order they arrived to that class.

Figure~\ref{fig:twoclass_example} shows an example with two classes of customer.

\begin{figure}
\begin{center}
\includestandalone[width=0.7\textwidth]{img/priority_queue}
\end{center}
\caption{An example of a two-class priority queue.}
\label{fig:twoclass_example}
\end{figure}

The key feature here is the $K \times K$ class change matrix
$\Theta = (\theta_{i,j})$. All elements $\theta_{i,j}$ where $i \neq j$ are
rates, and so are non-negative real numbers, if customers of class $i$ cannot
change to customers of class $j$ directly, then $\theta_{i,j} = 0$. The diagonal
values $\theta_{i,i}$ are unused as customers cannot change to their own class.
All elements $\theta_{i,i+1}$ represent the direct upgrade rates; all elements
$\theta_{i,i-1}$ represent the direct downgrade rates, while all other elements
can be thought of as `skip-grades`.
This is shown in Figure~\ref{fig:skipgrades}.

\begin{figure}
\begin{center}
\includestandalone[width=0.55\textwidth]{img/skipgrades}
\end{center}
\caption{Representations of parts of the matrix $\Theta$. Example when $K=5$.}
\label{fig:skipgrades}
\end{figure}


\subsection{Simulation Model Logic}

\subsection{Markov Chain Models}
The situation described in words in Section~\ref{sec:model} can be described
precisely as two different Markov chains.
The first, described in Section~\ref{sec:state_formulation}, describes the
overall changes in state, where a state records the number of customers of each
class at the node. This is useful for analysing system wide statistics such as
average queue size.
The second, described in Section~\ref{sec:sojourn_formulation}, describes how an
individual arriving customer experiences the system until their exit. This is
useful for analysing individual customers' statistics such as average sojourn
time.
Section~\ref{sec:bound} explores a bounded approximation for numerically
tractable analysis, and gives guidelines on choosing a large enough bound so as
to sufficiently approximate an unbounded system.
Section~\ref{sec:stationary} explores the parameter requirements for these
systems to have steady state distributions, and not grow infinitely.

\subsubsection{Discrete State Markov Chain Formulation}\label{sec:state_formulation}
Let
$\underline{\mathbf{s}}_t = (s_{0,t}, s_{1,t}, \dots, s_{K-1,t}) \in \mathbb{N}^K$
represent the state of the system at time step $t$, where $s_{k,t}$ represents
the number of customers of class $k$ present at time step $t$.

Then the rates of change between $\underline{\mathbf{s}}_t$ and
$\underline{\mathbf{s}}_{t+1}$ are given by Equation~\ref{eqn:transitions},
where $\underline{\mathbf{\delta}} = \underline{\mathbf{s}}_t - \underline{\mathbf{s}}_{t+1}$,

\begin{equation}\label{eqn:transitions}
q_{\underline{\mathbf{s}}_t, \underline{\mathbf{s}}_{t+1}} = 
\begin{cases}
\lambda_k & \text{if } \delta_k = 1 \text{ and } \delta_i = 0 \; \forall \; i \neq k \\
B_{k,t} \mu_k & \text{if } \delta_k = 1 \text{ and } \delta_i = 0 \; \forall \; i \neq k \text{ and } \sum_{i < k} s_{i,t} < c \\
(s_{k,t} - B_{k,t}) \theta_{k_0,k_1} & \text{if } \delta_{k_0} = -1 \text{ and } \delta_{k_1} = 1 \text{ and } \delta_i = 0 \; \forall \; i \neq k_0, k_1 \\
0 & \text{otherwise.}
\end{cases}
\end{equation}

and $B_{k,t}$, representing the number of customers of class $k$ currently in
service at time step $t$, is given by Equation~\ref{eqn:inservice}.

\begin{equation}\label{eqn:inservice}
B_{k,t} =\min\left(c - \min\left(\sum_{i < k} s_{i,t}, c\right), s_{k,t}\right)
\end{equation}

\subsubsection{Sojourn Time Markov Chain Formulation}\label{sec:sojourn_formulation}
Let $\underline{\mathbf{z}}_t = (z_{0,t}, z_{1,t}, \dots, z_{n,t} \dots, z_{K-1,t}, b_t, n_t) \in \mathbb{N}^{K+2}$
represent the state of a particular customer at time step $t$, where $n_t$
represents that customer's class at time $t$; $z_{k,t} \; \forall \; k < n$
represents the number of customers of class $k$ in front of the customer in the
queue at time $t$; $z_{k,t} \; \forall \; n < k < K$ represents the number of
customers of class $k$ behind the customer in the queue at time $t$; and $b_t$
represent the number of customers of class $n_t$ behind the customer in the
queue at time $t$.
Also let $\star$ represent an absorbing state, representing the state where that
customer has finished service and left the system.

Then the rates of change between $\underline{\mathbf{z}}_t$ and
$\underline{\mathbf{z}}_{t+1}$ are given by Equation~\ref{eqn:transitions_sojourn},
where $\underline{\mathbf{\delta}} = \underline{\mathbf{z}}_t - \underline{\mathbf{z}}_{t+1}$,

\begin{equation}\label{eqn:transitions_sojourn}
\resizebox{\textwidth}{!}{%
$q_{\underline{\mathbf{z}}_t, \underline{\mathbf{z}}_{t+1}} = 
\begin{cases}
\mu_n & \text{if } z_{t+1} = \star \text{ and } \sum_{k \leq n} z_{k, t} < c \\
\lambda_n & \text{if } \delta_K = 1 \text{ and } \delta_i = 0 \; \forall \; i \neq K \\
\lambda_k & \text{if } \delta_k = 1 \text{ and } \delta_i = 0 \; \forall \; i \neq k \text{ and } k \neq n\\
A_{k,n,t} \mu_k & \text{if } \delta_k = -1 \text{ and } \delta_i = 0 \; \forall \; i \neq k \text{ and } k < K\\
\tilde{A}_{n,t} \mu_n & \text{if } \delta_K = -1 \text{ and } \delta_i = 0 \; \forall \; i \neq K\\
(z_{k_0,t} - A_{k_0,n,t}) \theta_{k_0,k_1} & \text{if } \delta_{k_0} = -1 \text{ and } \delta_{k_1} = 1 \text{ and } \delta_i = 0 \; \forall \; i \neq k_0, k_1 \text{ and } k_0 < K \text{ and } k_1 \neq n, K, K+1 \\
(z_{K,t} - \tilde{A}_{n,t}) \theta_{n,k} & \text{if } \delta_K = -1 \text{ and } \delta_{k} = 1 \text{ and } \delta_i = 0 \; \forall \; i \neq k, n \text{ and } k < K \\
(z_{k,t} - A_{k,n,t}) \theta_{k,n} & \text{if } \delta_k = -1 \text{ and } \delta_K = 1 \text{ and } \delta_i = 0 \; \forall \; i \neq k, K \\
\theta_{n, k} & \text{if } \delta_n = z_{K,t} \text{ and } \delta_K = -z_{K,t} \text{ and } \delta_{K+1} = n - k \text{ and } \delta_i = 0 \text{ otherwise, and } \sum_{k \leq n} z_{k, t} < c \\
0 & \text{otherwise.}
\end{cases}$%
}
\end{equation}

and $A_{k,n,t}$ and $\tilde{A}_{n, t}$ are given by
Equations~\ref{eqn:inservice_adapt} and~\ref{eqn:inservice_adapt_tilde}.

\begin{equation}\label{eqn:inservice_adapt}
A_{k,n,t} =
\begin{cases}
\min\left(c, \sum_{i \leq k} z_{i,t}\right) - \min\left(c \sum_{i < k} z_{i,t}\right) & \text{if } k \leq n \\
\min\left(c, \sum_{i \leq k} z_{i,t} + 1 + z_{K,t}\right) - \min\left(c \sum_{i < k} z_{i,t} + 1 + z_{K,t}\right) & \text{if } n < k < K
\end{cases}
\end{equation}

\begin{equation}\label{eqn:inservice_adapt_tilde}
\tilde{A}_{n,t} =
\min\left(c, \sum_{i \leq n} z_{i,t} + 1 + z_{K,t}\right) - \min\left(c, \sum_{i \leq n} z_{i,t} + 1\right) \\
\end{equation}


The expected time to absorption can be calculated from each state.
Customers arrive in all states where $z_{K,t} = 0$, and their class can be
determined by $n$. Combining these times to absorption with the state
probabilities found in the previous section, the sojourn times for each customer
class can be found.


\subsubsection{Bounded Approximation}\label{sec:bound}

\subsubsection{Existence of Stationary Distributions}\label{sec:stationary}
A key difference between using a bounded approximation to analysing infinite
Markov chains is the existence of stationary distributions. All bounded
approximations have steady state distributions, even if the corresponding
infinite Markov chain does not, leading to spurious results in these cases.
Theorem~\ref{thrm:steadystate} gives a naive check for the existence or
non-existence of steady states, but does not cover all possibilities.

\begin{theorem}\label{thrm:steadystate}
For $M/M/c$ work conserving queue with $K$ classes of customer, with arrival
rate $\lambda_k$ for customers of class $k$ and service rate $\mu_k$ for
customers of class $k$:
\begin{enumerate}
  \item if $\sum_i \lambda_i < c \min_i \mu_i$ then it will reach steady state,
  \item if $\sum_i \lambda_i > c \max_i \mu_i$ then it will never reach steady state.
\end{enumerate}
\end{theorem}

Note that this result does not assume any particular service discipline such as
first-in-first-out or prioritised classes, but holds for any work conserving
discipline.

\begin{proof}
The queue will reach steady state if the rate at which customers are added to
the queue is less than the rate at which customers leave the queue.
As arrivals are not state dependent, customers are added to the queue at a rate
$\sum_i \lambda_i$ when in any state.
The rate at which customers leave the queue is state dependent, depending on the
service discipline.

We do not need to consider cases when there are less than $c$ customers present,
as here any arrival will increase the rate at which customers leave the queue,
as that arrival would enter service immediately.
Considering the cases where there are $c$ or more customers in the queue, there
are two extreme cases, either:

\begin{enumerate}
  \item All customers in service are of the class with the slowest service rate.
  In this case the rate at which customers leave the queue is $c \min_i \mu_i$,
  which is the slowest possible rate at which customers can leave the queue.
  If $\sum_i \lambda_i < c \min_i \mu_i$ then the rate at which customers enter
  the queue is smaller than the smallest possible rate at which customers leave
  the queue, and so will always be smaller than the rate at which customers
  leave the queue in all states. Therefore the system will reach steady state.
  \item All customers in service are of the class with the fastest service rate.
  In this case the rate at which customers leave the queue is $c \max_i \mu_i$,
  which is the fastest possible rate at which customers can leave the queue.
  If $\sum_i \lambda_i > c \max_i \mu_i$ then the rate at which customers enter
  the queue is greater than the largest possible rate at which customers leave
  the queue, and so will always be greater than the rate at which customers
  leave the queue in all states. Therefore the system cannot reach steady state.
\end{enumerate}
\end{proof}

If $c \min_i \mu_i < \sum_i \lambda_i < c \max_i \mu_i$ then more investigation
is needed.

\section{Effect of $\Theta$ on System Behaviour}

\bibliographystyle{plain}
\bibliography{refs}

\end{document}
