\documentclass{article}
\usepackage{fullpage}
\usepackage{parskip}
\usepackage{standalone}
\usepackage{graphicx}
\usepackage{booktabs}
\usepackage{subcaption}
\usepackage{hyperref}
\usepackage{amsmath}
\usepackage{amsfonts}
\usepackage[table]{xcolor}
\usepackage{tikz}
\usetikzlibrary{arrows}
\usetikzlibrary{decorations.pathreplacing}


\title{Dynamic Priority Classes (?)}
\author{authors (?)}
\date{}

\begin{document}
\maketitle

\section{The Model}
Here we consider an $M/M/c$ queue with $K$ classes of customer.
Order and label the customer classes $0, 1, 2, \dots, k$, with customer classes
with lower labels having priority over customer classes of higher labels.
The index $k$ will be used to represent customer classes.
Let:

\begin{itemize}
  \item $\lambda_k$ be the arrival rate of customers of class $k$,
  \item $\mu_k$ be the service rate of customers of class $k$,
  \item $\theta_{k_ik_j}$ be the rate at which customers of class $k_i$ change
  to customers of class $k_j$.
\end{itemize}

Figure~\ref{fig:twoclass_example} shows an example with two classes of customer.

\begin{figure}
\begin{center}
\includestandalone[width=0.7\textwidth]{img/priority_queue}
\end{center}
\caption{An example of a two-class priority queue.}
\label{fig:twoclass_example}
\end{figure}


\section{Markov Chain Formulation}
Let $\underline{\mathbf{s}}_t = (s_0t, s_1t, \dots, s_Kt) \in \mathbb{R}^K$
represent the state of the system at time step $t$, where $s_kt$ represents the
number of customers of class $k$ present at time step $t$.

Then the rates of change between $\underline{\mathbf{s}}_t$ and
$\underline{\mathbf{s}}_{t+1}$ are given by Equation~\ref{eqn:transitions},
where $\underline{\mathbf{\delta}} = \underline{\mathbf{s}}_t - \underline{\mathbf{s}}_{t+1}$,

\begin{equation}\label{eqn:transitions}
q_{\underline{\mathbf{s}}_t, \underline{\mathbf{s}}_{t+1}} = 
\begin{cases}
\lambda_k & \text{if } \delta_k = 1 \text{ and } \delta_i = 0 \; \forall \; i \neq k \\
\min(s_kt, c) \mu_k & \text{if } \delta_k = 1 \text{ and } \delta_i = 0 \; \forall \; i \neq k \text{ and } s_{it} = 0 \; \forall \; i < k \\
W_{kt} \theta_{k_0k_1} & \text{if } \delta_{k_0} = -1 \text{ and } \delta_{k_1} = 1 \text{ and } \delta_i = 0 \; \forall \; i \notin (k_0, k_1)
\end{cases}
\end{equation}

and $W_{kt}$, representing the number of customers present but not in service at
time step $t$, is given by Equation~\ref{eqn:waiting}.

\begin{equation}\label{eqn:waiting}
W_{kt} = s_{kt} - \min\left(c - \sum_{i < k} s_{it}, s_{kt}\right)
\end{equation}


\bibliographystyle{plain}
\bibliography{refs}



\end{document}