\documentclass{article}
\usepackage{fullpage}
\usepackage{parskip}
\usepackage{standalone}
\usepackage{graphicx}
\usepackage{booktabs}
\usepackage{subcaption}
\usepackage{hyperref}
\usepackage{amsmath}
\usepackage{amsfonts}
\usepackage[table]{xcolor}
\usepackage{tikz}
\usetikzlibrary{arrows}
\usetikzlibrary{decorations.pathreplacing}


\title{Dynamic Priority Classes (?)}
\author{authors (?)}
\date{}

\begin{document}
\maketitle

\section{The Model}
Here we consider an $M/M/c$ queue with $K$ classes of customer.
Order and label the customer classes $0, 1, 2, \dots, K-1$, with customer
classes with lower labels having priority over customer classes of higher
labels. The index $k$ will be used to represent customer classes.
Let:

\begin{itemize}
  \item $\lambda_k$ be the arrival rate of customers of class $k$,
  \item $\mu_k$ be the service rate of customers of class $k$,
  \item $\theta_{k_i,k_j}$ be the rate at which customers of class $k_i$ change
  to customers of class $k_j$.
\end{itemize}

Figure~\ref{fig:twoclass_example} shows an example with two classes of customer.

\begin{figure}
\begin{center}
\includestandalone[width=0.7\textwidth]{img/priority_queue}
\end{center}
\caption{An example of a two-class priority queue.}
\label{fig:twoclass_example}
\end{figure}


\section{State Markov Chain Formulation}
Let
$\underline{\mathbf{s}}_t = (s_{0,t}, s_{1,t}, \dots, s_{K-1,t}) \in \mathbb{R}^K$
represent the state of the system at time step $t$, where $s_{k,t}$ represents
the number of customers of class $k$ present at time step $t$.

Then the rates of change between $\underline{\mathbf{s}}_t$ and
$\underline{\mathbf{s}}_{t+1}$ are given by Equation~\ref{eqn:transitions},
where $\underline{\mathbf{\delta}} = \underline{\mathbf{s}}_t - \underline{\mathbf{s}}_{t+1}$,

\begin{equation}\label{eqn:transitions}
q_{\underline{\mathbf{s}}_t, \underline{\mathbf{s}}_{t+1}} = 
\begin{cases}
\lambda_k & \text{if } \delta_k = 1 \text{ and } \delta_i = 0 \; \forall \; i \neq k \\
B_{k,t} \mu_k & \text{if } \delta_k = 1 \text{ and } \delta_i = 0 \; \forall \; i \neq k \text{ and } \sum_{i < k} s_{i,t} < c \\
(s_{k,t} - B_{k,t}) \theta_{k_0,k_1} & \text{if } \delta_{k_0} = -1 \text{ and } \delta_{k_1} = 1 \text{ and } \delta_i = 0 \; \forall \; i \neq k_0, k_1 \\
0 & \text{otherwise.}
\end{cases}
\end{equation}

and $B_{k,t}$, representing the number of customers of class $k$ currently in
service at time step $t$, is given by Equation~\ref{eqn:inservice}.

\begin{equation}\label{eqn:inservice}
B_{k,t} =\min\left(c - \min\left(\sum_{i < k} s_{i,t}, c\right), s_{k,t}\right)
\end{equation}


\section{Sojourn Time Markov Chain Formulation}
Let $\underline{\mathbf{z}}_t = (z_{0,t}, z_{1,t}, \dots, z_{n,t} \dots, z_{K-1,t}, b_t, n_t) \in \mathbb{R}^{K+2}$
represent the state of a particular customer time step $t$, where $n_t$
represents that customer's class at time $t$; $z_{k,t} \; \forall \; k < n$
represents the number of customers of class $k$ in front of the customer in the
queue at time $t$; $z_{k,t} \; \forall \; n < k < K$ represents the number of
customers of class $k$ behind the customer in the queue at time $t$; and $b_t$
represent the number of customers of class $n_t$ behind the customer in the
queue at time $t$.
Also let $\star$ represent an absorbing state, representing the state where that
customer has finished service and left the system.

Then the rates of change between $\underline{\mathbf{z}}_t$ and
$\underline{\mathbf{z}}_{t+1}$ are given by Equation~\ref{eqn:transitions_sojourn},
where $\underline{\mathbf{\delta}} = \underline{\mathbf{z}}_t - \underline{\mathbf{z}}_{t+1}$,

\begin{equation}\label{eqn:transitions_sojourn}
\resizebox{\textwidth}{!}{%
$q_{\underline{\mathbf{z}}_t, \underline{\mathbf{z}}_{t+1}} = 
\begin{cases}
\mu_n & \text{if } z_{t+1} = \star \text{ and } \sum_{k \leq n} z_{k, t} < c \\
\lambda_n & \text{if } \delta_K = 1 \text{ and } \delta_i = 0 \; \forall \; i \neq K \\
\lambda_k & \text{if } \delta_k = 1 \text{ and } \delta_i = 0 \; \forall \; i \neq k \text{ and } k \neq n\\
A_{k,n,t} \mu_k & \text{if } \delta_k = -1 \text{ and } \delta_i = 0 \; \forall \; i \neq k \text{ and } k < K\\
\tilde{A}_{n,t} \mu_n & \text{if } \delta_K = -1 \text{ and } \delta_i = 0 \; \forall \; i \neq K\\
(z_{k_0,t} - A_{k_0,n,t}) \theta_{k_0,k_1} & \text{if } \delta_{k_0} = -1 \text{ and } \delta_{k_1} = 1 \text{ and } \delta_i = 0 \; \forall \; i \neq k_0, k_1 \text{ and } k_0 < K \text{ and } k_1 \neq n, K, K+1 \\
(z_{K,t} - \tilde{A}_{n,t}) \theta_{n,k} & \text{if } \delta_K = -1 \text{ and } \delta_{k} = 1 \text{ and } \delta_i = 0 \; \forall \; i \neq k, n \text{ and } k < K \\
(z_{k,t} - A_{k,n,t}) \theta_{k,n} & \text{if } \delta_k = -1 \text{ and } \delta_K = 1 \text{ and } \delta_i = 0 \; \forall \; i \neq k, K \\
\theta_{n, k} & \text{if } \delta_n = z_{K,t} \text{ and } \delta_K = -z_{K,t} \text{ and } \delta_{K+1} = n - k \text{ and } \delta_i = 0 \text{ otherwise, and } \sum_{k \leq n} z_{k, t} < c \\
0 & \text{otherwise.}
\end{cases}$%
}
\end{equation}

and $A_{k,n,t}$ and $\tilde{A}_{n, t}$ are given by
Equations~\ref{eqn:inservice_adapt} and~\ref{eqn:inservice_adapt_tilde}.

\begin{equation}\label{eqn:inservice_adapt}
A_{k,n,t} =
\begin{cases}
\min\left(c - \min\left(\sum_{i < k} z_{i,t}, c\right), z_{k,t}\right) & \text{if } k \leq n \\
\min\left(c - \min\left(1 + \sum_{i < k} z_{i,t}, c\right), z_{k,t}\right) & \text{if } n < k < K
\end{cases}
\end{equation}

\begin{equation}\label{eqn:inservice_adapt_tilde}
\tilde{A}_{n,t} =
\min\left(c - \min\left(1 + \sum_{i \leq n} z_{i,t}, c\right), z_{K,t}\right) \\
\end{equation}

The expected time to absorption can be calculated from each state.
Customers arrive in all states where $z_{K,t} = 0$, and their class can be
determined by $n$. Combining these times to absorption with the state
probabilities found in the previous section, the sojourn times for each customer
class can be found.

\bibliographystyle{plain}
\bibliography{refs}


\end{document}